\documentclass{article}
\usepackage{graphicx}
\usepackage{amssymb}
\usepackage{amsmath}
\usepackage{tikz}
\usepackage{url}
\usetikzlibrary{shapes}


\linespread{1.4}

\title{Homework - 1}

\author{Names Here}



\begin{document}

\maketitle

\clearpage


\section{Experience with GenBank/Entrez database}
There are 73 results for the Myoglobin gene\footnote{\url{http://www.ncbi.nlm.nih.gov/gene?term=myoglobin}}. There are so many results because many different organisms have the same gene, but the sequence of that gene might differ amongst them.
The results inculde sequences of the gene as found in \textit{Homo Sapiens} (Humans), \textit{Equus ferus caballus} (Horse), \textit{Perkinsus marinus} (A prevalent pathogen of oysters), \textit{Rattus norvegicus} (Rat), amongst other organisms.\\
\\
Searching the entire GenBank database for the keyword 'Human Genome', gave us results matching under categories like Nucleotides, Genes, Chromosomes, Genome, etc. On following the 'Genome' category, we ultimately ended up with the Human Genome page (\url{http://www.ncbi.nlm.nih.gov/genome/guide/human/}), where we could access the sequence for all the human chromosomes. For each human chromosome, we could see the list of genes which were associated with regions of the chromosomes, their utility with respect to certain genes and proteins, and options to view and download the data in different formats.\\
\\
In all 37 eukaryotic genome sequencing projects are complete\footnote{\url{http://www.ncbi.nlm.nih.gov/genomes/leuks.cgi}}, of which are, 3 mammalian genome sequences (\textit{Homo Sapiens} (Human), \textit{Mus Musculus} (House Rat), \textit{Pan troglodytes verus} (West African Chimpanzee)), 5 plant genome sequences, and 16 fungii sequences. 2702 Viral genome sequencing projects are complete\footnote{\url{http://www.ncbi.nlm.nih.gov/genomes/GenomesGroup.cgi?taxid=10239&opt=Virus}}. Among single celled organizms (Prokaryotes), 1702 genome sequencing projects are complete\footnote{\url{http://www.ncbi.nlm.nih.gov/genomes/lproks.cgi}}, of which 1643 are bacterial genome sequences, and the rest are Archea. Thus, in all 4441 organisms and virii have had their genomes completely sequenced.

The above counts reflect the sequencing projects which GenBank reports as 'Completed'. There are certain organisms, like the Cow, whose sequencing projects are reported to be complete in the media but GenBank does agree with that yet. So they have not been taken into account.
\clearpage

\section{Experience with PubMed}
Stevens Johnson syndrome (SJS) is a skin condition where cell death occurs resulting in the separation of the epidermis from the dermis. This results in lesions on the skin. SJS affects the eyes and the eyelids in many cases. It is believed that SJS is caused mainly due to reactions to certain kinds of medication. Along with this, certain types of infections and, sometimes, cancers are also thought to be causes of SJS. Medication seems to be the leading cause for SJS and related complications. According to one of the studies, antimicrobial drugs seem to account for almost half the cases of SJS. Next in line are non-steroidal anti-inflammatory drugs followed by anti-seizure drugs. Nevirapine, which is used to contain HIV, is one of the leading causes of SJS. It accounts for almost a third of all cases. In some cases, it was also hypothesized that the seasonal influenza vaccine with another drug was the cause of SJS in a patient. Sulfasalazine was also found to be a trigger in a couple of documented cases. 
As far as treatment of SJS goes, there doesn't seem to be any consensus on the preferred mode of treatment. Usually corticosteroids and intravenous immunoglobulin (IVIG) are being used to treat SJS. Although treatment with IVIG is mainly thought to be beneficial, recent studies have cast doubts on this conclusion. Plasmapheresis and cyclosporine have also been effectively, but there isn't a lot of literature on the efficacy of these two drugs. Amniotic membrane transplant is another approach that is used to treat patients with SJS. \\
\\
The Echidna is an egg-laying mammal found mainly in Australia. Their body is covered in coarse hair and spines. They are believed to have evolved from the platypus and their ancestors were amphibians. They have long beaks with which they catch their prey. According to one study, the short beak echidna, Tachyglossus aculeatus, can live for as long as 50 years. This is almost 3.7 times its predicted life-span based on its body mass. This is believed to be due to a peroxidation-resistant membrane composition. Their reproduction, too, is somewhat of a mystery. One study found that the females repeatedly go into hibernation even after mating. Males were found to mate with females that were in torpor. It is believed that this is due to extreme competition between highly promiscuous males. \\
\\
Professor Skiena has 27 papers listed on Pubmed as of 10/02/2011. Quite a few papers concentrate on core computational biology areas such as PCR primer desgin, virus attenuation etc. and have a considerable number of citations. This wouldn't be the case if the papers weren't up to the mark :) All of the papers on PubMed were also available on other sources like Google Scholar. 
\clearpage

\section{Problem 3}
\clearpage

\section{Longest Overlap of Strings}
Given two strings, $A$, and $B$, we want to compute the longest overlap of the two strings, which is the length of the longest suffix of $A$, which also is a prefix of $B$. Let us assume that $A$ is of length $n$ and $B$ is of length $m$.

\begin{enumerate}
\item {Construct the suffix tree of A and B, concatenated with a \# separating them. This can be done in time linear in the length of the string \footnote{This can be done by using one of the different linear time suffix tree construction algorithms available, given by Ukkonen, McCreight, and Weiner.}, i.e. $O(n+m)$ }
\item {Start traversing the tree starting from the first character of $B$. If there is a valid overlap between the two strings, then there would be a suffix of the concatenated string, of the form, X\#XY, where $X$ is the desired overlap string, and $Y$ is the rest of the string $B$. Thus if we traverse the suffix tree starting with the first character of $B$, we will have the same structure, till we encounter the \# character. If there is no overlap, we would not encounter the \#, and reach the end of the tree.}
\item {Initialize a counter, and keep incrementing the counter and traversing down the tree by following the characters of B. When, we encounter the \# character, the counter value is the length of the longest overlap. If we reach the end of the tree without encountering the \#, the strings have no overlap.} 
\end{enumerate}

Thus we can find the length of the longest overlap, in time $O(n+m)$.
\clearpage

\section{Bio Palindromes}
Bio-palindromes are different from the traditional string palindromes such as ABCDCBA in the sense that in a bio-palindrome, the reverse of the string is a complement of the actual string instead of an exact copy. So, a string like ACGT is a bio-palindrome. \\
\\
We can use the existing algorithm to find the longest palindrome to also find the longest bio-palindrome in a string. In the original algorithm, we start off from a pivot element (in case of odd length palindromes) or with no character (in case of even length palindromes) and keep expanding to the left as well as to the right, checking if the characters at each step exactly match each other. \\
\\
The only difference that we need to find bio-palindromes it that we should check if the characters are complements of each other. We can create a hashmap mapping each character to its complement. For example, map['A'] = 'T'. We do the same for G, C and T. Then we use the original algorithm and instead of checking if the character on the left matches the one on the right, we check if the one on the left matches the hashmap value of the one on the right. 
\clearpage

\section{Problem 6}
(a) Does not work. The example string itself “ABBABABA” is a 
counter-example. The greedy strategy would select the largest 
palindrome “BABAB”, and break the string into A-B-BABAB-A, 
which is four parts. The optimal solution is ABBA-BAB-A, or ABBA-B-ABA.\\
\\
(b) Does not work. “ABABABA” is a counter-example. The greedy strategy 
would select the largest palindromes from left. The largest and left-most
palindrome is `BABA' which starts at the 2nd position. This leaves, `A' and
`BA' at either ends, which leads to partitioning the string in this fashion
`A-BABA-B-A', i.e. three cuts. Whereas, the optimal partition would have
been `ABA-BABA', with only one cut.\\
\\
(c) Here we present an $O(N^2)$ solution to the problem. \\

Given a string $s$ of length $N$, for each position $i$, we need to know the lengths of different palindromes that start at that position. We do a bit of preprocessing to get this information. The time complexity of this preprocessing is $O(n^2)$. We do this by using the algorithm for finding the longest palindrome in a string. Given a position $i$, we can compare characters equidistant to its left and right and keep increasing the distance till these equidistant characters match. For each position, we maintain a set which stores all the unique lengths of palindrome starting at that position. We can find odd as well as even length palindromes this way. For each position $i$, there can be a maximum of $N$ entries. \\
\\
Now, for each position $i$, we have a set which contains the unique lengths of all palindromes starting at that position. Now we can use a recursive function with memoization to find the partition of the string into the minimum number of palindromes. 
\\
Code : 
\begin{verbatim}
// v[i] is a set of integers that stores the lengths of all 
// palindromes starting at position i. The value '1' is always in this set.
vector<set<int> > v; 

// MAXN is the maximum length of the string. dp[i] stores the minimum 
// number of partitions required for the string S[i..N-1];
int dp[MAXN]; 

// We initialize all values in dp[] to -1 before calling solve(0).
#define INFINITY (int)1e9
int solve(int position)
{
  if(position == N)
    return 0; // We don't need any more partitions if we reach the end of the string
	
  int &ret = dp[position];
  if(ret != -1)
    return ret; // If we have seen this value before, don't recalculate.
	
  ret = INFINITY;

  set<int> s = v[position];
  for(set<int> :: iterator it = s.begin(); it != s.end(); it++)
  {
    ret = min(ret, solve(position + *it) + 1);
  }
  return ret;
}
\end{verbatim}
The complexity of calculating all the values is $O(N^2)$. This is because there are at most $N$ states that we will visit and for each of the states, there are $N$ other possible states. 

So, the complexity of both, preprocessing and calculating, is $O(N^2)$. So, the solution has a complexity of $O(N^2)$ too. 

\clearpage

\end{document}
